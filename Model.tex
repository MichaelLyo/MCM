\documentclass[a4paper,11pt]{article}
%在此可进行页面设置
%\textwidth=14cm
%\textheight=22cm
\usepackage{tikz}
\usetikzlibrary{mindmap,trees}
\usepackage{verbatim}
\usepackage{float}
\usepackage[section]{placeins}
\usepackage{setspace}%使用间距宏包
\usepackage[76620]{MCMPackage}  %队号在这里填写
%\usepackage[XXX,nosheet]{MCMPackage}%这个参数形式可去掉summary sheet首页。
\problem{C}  %选题

\title{Energy Production}%在此插入论文标题
\date{\today}

%设置段落之间的距离,若不需要删除或者注释掉即可。
\setlength\parskip{.5\baselineskip}

%为了首行缩进
%\usepackage{indentfirst}
%\setlength{\parindent}{2em}

\makeatletter
\def\@cite#1#2{\textsuperscript{[{#1\if@tempswa , #2\fi}]}}
\makeatother

%设置参考文献的小上标

\begin{document}
%摘要
\begin{abstract}

\par For the sake of \textbf{\emph{Measuring the Evolution and Influence in Society's Information Networks}}, our idea is to set up a comprehensive model with plentiful properties of nodes which are consistent with the current states. Therefore, the concept of \textbf{\emph{Complex Network}} is introduced to depict information dissemination vividly. What's more, the \textbf{\emph{Twitter's Network Dataset}} is utilized to verify most of the models we build.


\par About Problem A, in order to \textbf{\emph{Explore the Information Flow}} and \textbf{\emph{Filter out Valuable Information}}, we establish the \textbf{\emph{Information Flow and Filter Model}}. Having considered the diversity in information media, the \textbf{\emph{Flexible Network Structure}} is explored to suit for the various types of information flow. Meanwhile, to gain a better information filter result, we create a filter consisting of ~$6$~ evaluation indexes. After theoretical preparation, we use the \textbf{\emph{Twitter's Network Dataset}} to establish Internet Information Flow and Filter Model for simulation. Finally, the coverage rate of information dissemination is equal to ~$96.7\%$~, after reaching the steady state of information dissemination.


\par As to Problem B \& C, based on the Information Dissemination Model, our assignment is to use the data to verify the \emph{\textbf{Reliability of the Model}} and \emph{\textbf{Predict the Future Data}}. We establish the \emph{\textbf{Network Evolution Model}} to imitate the dynamic network structure in a single medium. The \textbf{\emph{Average Euclidean Distance}} of the \textbf{\emph{real data}} and the \emph{\textbf{predicted data}} is ~$1.784$~, which stands for a good forecast result. After that, we take the model as the basis for the prediction of the Society's Information Network in $2050$. Without considering the emergence of new technologies, the cell phone usage is 51.34\%, which is the highest in the $4$ kinds of media. However, after making bold assumptions that the \emph{\textbf{Quantum Communication}} would be invented in $2035$, the share of the information medium will change dramatically, and the \emph{\textbf{Quantum Communication }} would be the main medium.

\par  When solving the Problem D \& E, we are required to analyze the Public Opinion Dissemination. We divide the network nodes into $3$ types: \textbf{\emph{Individuals, Opinion Leaders and Internet Marketer Group}}. And then we establish the \emph{\textbf{Public Opinion Dissemination Model}} to simulate the real situation of the public opinion dissemination, and draw the contrast curves of public opinion dissemination coverage rate before and after \textbf{\emph{Taking "PUSH" Operation}}. Finally, we screen out the effect factors of public opinion dissemination, and use the example of \textbf{\emph{President Obama's Nearly Opened Twitter}} to verify the stability of the model.


\par The accuracy and stability of our model are proved by simulation using the \emph{\textbf{Twitter's Network Dataset}}. Furthermore, the Information Dissemination Network Model can adapt to different types of media. By slightly adjusting the coefficients, we can apply this network model in various fields of information dissemination.


\textbf{Keywords: Information Dissemination Model; Dynamic Network Structure; Twitter's Network Dataset; Simulation \& Verification} 

\end{abstract}
\maketitle%插入标题
\thispagestyle{empty}%本页不遍页码

\newpage%另起一页插入目录
\thispagestyle{empty}%防止目录两页而出现一页带页码
\begin{spacing}{0.5}%%行间距变为single-space
\tableofcontents%目录
\maketitle
\thispagestyle{empty}
\end{spacing}

\newpage%另起一页书写正文
\pagenumbering{arabic}%开始编页码(阿拉伯数字)

\section{Introduction}
%In reference \cite{RefB}.%引用参考文献

\subsection{Background}
 
\par Nowadays, with the development of technology and the deterioration of environment, many countries, especially the Uinted States, began to value the clean and renewable energy, such as natural gas and electricity. Meanwhile, traditional energy, such as gasoline and coal, still plays an important role in economics. And it is obvious that the usage of the clean and renewable energy will support the future of mankind.
\par In the United States, an interstate compact is an agreement between two or more states. Frequently, these agreements create a new governmental agency which is responsible for administering or improving some shared resource such as a seaport or public transportation infrastructure.\cite{1} In 1970, the Western Interstate Energy Compact was formed, which was devoted to foster cooperation to develop and manage the nuclear technology. Recently, California, Arizona, New Mexico, and Texas wish to form a new interstate energy compact focused on increased usage of cleaner, renewable energy sources. 
\par A data file provides 50 years of data in 605 variables on each of these four states' energy production and consumption, along with some demographic and economic information. In order to inform a set of goals for the new interstate energy compact, we will analysize the data file and model.





\subsection{Problem Restatement and Analysis}
%这一部分写问题分析以及我们要解决的问题
\subsubsection{Part \uppercase\expandafter{\romannumeral1}}
\par For Problem A, in order to lay the foundation of analyzing , we are required to create an energy profile for each of the four states. Meanwhile, the profiles should be created on the analysis of the given data.

\par In regard with Problem B, 

\par For problem C, 

\par For Problem D, in order to predict energy profile of usage of each state in 2025 and 2050, we need to base on the results of Problem A, B,C and the given data to establish regression equations or other models.

\subsubsection{Part \uppercase\expandafter{\romannumeral2}}
\par It is suggested in the Problem A that,

\par As for Problem B, 

\subsubsection{Part \uppercase\expandafter{\romannumeral3}}
\par On the basis of the analysis and models in the two parts before, we prepare the state profiles of 2009, our predictions towards energy usage and some recommendations on adopting goals for the energy compact.


\section{Assumption}
\begin{enumerate}%[(1)]
\renewcommand{\labelenumi}{(\theenumi)}
    \item In factor analysis, we suppose the corresponding valuables as the main factors when cumulative percentage is over 0.65 .
    \item When we do regression analysis, we suppose that F-test or t-test is passed if the Pr of them is less than 0.05.
    \item When we do regression analysis, we suppose that effect of regression is good if $R^{2}$ is greater than 0.65.
\end{enumerate}

%这里写模型假设


\section{Symbol Description}
%符号说明
In the section, we use some symbols for constructing the model as follows.

\begin{center}
\begin{tabular}{cc}%r 表示表格内文本右对齐;c表示居中对齐;l表示左对齐;| 表示竖铅直线
    \toprule[2pt]
    \textbf{Symbol} & \makecell[c]{\textbf{Description}}\\
    \hline
$R$&The correlation coefficient matrix of natural gas, electricity, fuel oil and coal coke.\\ 
$\lambda_i$&The eigenvalue of R, i=1,2,3,4.\\
$U$&The Eigenvector matrix for the eigenvalues of R.\\ 
$A$&The factor loading matrix.\\
$f_i$&Factor, i=1,2.\\ 
$w_1$&The weight of natural gas.\\ 
$w_2$&The weight of electricity.\\ 
$w_3$&The weight of fuel oil.\\ 
$w_4$&The weight of coal coke.\\ 
$\eta_1$&The proportion of renewable energy in total energy.\\
$\eta_2$&The proportion of different sector in renewable energy usage.\\
$Y$&The level of renewable energy usage.\\ 
$X_i$&Transportation(i=1), commerce(i=2), electricity(i=3), industry(i=4), resident(i=5), i=1,2,3,4,5.\\ 
$C,B_i$&The pairwise comparison matrixs, i=1,2,3,4,5.\\
$\beta,\beta_i$&weight vectors, i=1,2,3,4,5.\\ 
% $T_n$&The discrete time series of the node checking information.\\ 
% $p_{in}$& The probability of ~$Node~i$~ to check information at the moment of $T_n$.\\
% $q_{in}$&The probability that information is able to continue to spread\\
% $\lambda_i$&The probability of node's retransmission or comments after receiving the message.\\
% $N_n$& The number of nodes that have received information at the moment ~$T_n$~\\
% $Q_n$& The number of nodes that have received information at the moment ~$T_n$~\\
% $\gamma_i$& The new attention degree of a certain node in the unit time ~$\Delta t$~\\
% $D_i$&The degree of ~$Node~i$~\\
% $C_i$&The clustering of ~retransmission~$Node~i$\\
% $L_{ij}$&The shortes path length of ~$Node~i$~ and ~$Node~j$\\
    \bottomrule[2pt]
    %以下两条指令此处不用
    %\caption{表格标题}
    %\label{标签名}%便于交叉引用
\end{tabular}
\end{center}



\emph{P.S: Other symbol instructions will be given in the paper.}


% 4
\section{Energy Analysis Model}

% \par We divide 
% \par Factor Analysis is a multivariate statistical analysis method, which is based on the internal dependencies of the correlation matrix of the indicator. It attributes some of the variables that have complex relationship to a few irrelevant factors.\cite{2} We will use this method to deal with problem A in the section.

% Problem A: Energy profile for each of the four states
% 4.1
\subsection{Distinction approach of energy construction}
% 4.1.1
\subsubsection {MSN character analysis}

\par By observing the given data and acquiring knowledge from Appendix A. Mnemonic Series Names (MSN)\cite{3}, we can conclude that the MSNs are usually five-character codes, and the three parts of the characters generally represent the following meanings.

\begin{itemize}
    \item First and second characters: describes an energy source (for example, NG for natural gas, MG for motor gasoline);
    \item Third and fourth characters: describes an energy sector or an energy activity (for example, RC for residential consumption, PR for production);
    \item Fifth character: describes a type of data (for example, P for data in physical unit, B for data in billion Btu).	\cite{3}
\end{itemize}

\par As described above, the first part and the second part of the MSN code represent for certain specific meanings. However, in our analysis process, we are more concerned about the general classification of these MSN codes. 

\begin{figure}[h]%[!hptb]
    \centering 
    \includegraphics[width=0.8\textwidth]{./Pic/msn_5.png}
    % 图片标题
    \caption{The meaning of a MSN code, emphasizing the fifth character. Reference \url{https://www.eia.gov/state/seds/sep_use/notes/use_a.pdf}}
    \label{fig:msn_5}  
\end{figure}
\par We also found that the fifth character has six values at most, which can be seen in figure \ref{fig:msn_5}. Thus we start to focus more on the analysis of the fifth character. After comparing the units of the six and calculating, we find the relationship between these six symbols are
\begin{equation}
    B=KP~~~~~,~~~~~V=BD / 1000
\end{equation}

\par From the equation we can get the conclusion that B is a key factor between the two equations and can build a connection between them. And we find that all MSNs whose fifth character is B have the same unit Billion Btu, which can be an excellent representative of the MSNs who have the same prefix. So we filter the 605 MSNs and find 215 MSNs' suffix is B. At this stage we have reduced the 605 kinds of MSNs into 215. Every kind of the 215 MSNs contains several MSNs with different suffix but the same prefix.


% 4.1.2
\subsubsection {MSN classification analysis by energy unit}



\par In order to classify the 215 MSNs more clearly, we look into everyone of the division and find that a large percent of the kinds have the suffix P, which represents the standardized physical units. Physical units naturally represents for the energy classification in the real physical world, and there are mainly \textbf{4 kinds of standardized physical units} in the data set as can be seen in figure \ref{fig:classification_by_unit}, so it is reasonable to classify the data into 4 groups according to the units.
\par According to the analysis above, we suppose the four units as the following representatives.
\par \textbf{fule oil}: Thousand barrels;
\textbf{coal coke}: Thouesand short tons;
\textbf{electricity}: Million kilowatthours;
\textbf{natural gas}: Million cubic feet.

\begin{figure}[!h]%[!hptb] 
    \centering 
    \includegraphics[width=0.8\textwidth]{./Pic/classification_by_unit.png}
    \caption{Classification by unit in the suffix P.}
    \label{fig:classification_by_unit}  
\end{figure}

% 4.2
\subsection{Factor analysis function}

\par There are many relationships among MSNs. To simplify the complex dependencies between the four energy classification, we used factor analysis to establish an optimal model to analyze the actual weights of the four types of energy sources.

% \par According to the Formula in the Table A1.Consumption Variables,\cite{4} we remove variables that can be obtained by addition and subtraction of other variables, so that we can avoid the repeat of variables when we calculate the correlation coefficient matrix. 
\begin{enumerate}[Step 1:]
\item Standardize data samples.

\item Calculate the correlation matrix R of the sample.

\item Find the eigenvalues and eigenvectors of the correlation matrix R.

\item According to the cumulative contribution rate of the system to determine the number of the main factor.

\item Calculate the factor load matrix A.

\item Determine the factor model.

\item Based on the above calculation results, the system is analyzed.
\end{enumerate}

\par By using the data processed, we obtain R, which is the correlation coefficient matrix of natural gas, electricity, fuel oil and coal coke.
\begin{equation}
    \centering
R=\begin{pmatrix} 1 & -0.389890 & 0.432474 & -0.004734 \\ -0.389890 & 1 & -0.094940 &-0.158034 \\ 0.432474 & -0.094940 & 1 & -0.016881 \\ -0.004734 & -0.158034 & -0.016881 & 1\end{pmatrix}
\end{equation}


\par Then we can obtain 
\[
\lambda_1=1.638729, \lambda_2=0.454611, \lambda_3=1.089176, \lambda_4=0.817484
\]
which are the eigenvalue of R.
\par Meanwhile, we obtain the eigenvector matrix for the eigenvalues of R.
\begin{equation}
    \centering
U=\begin{pmatrix} -0.666575 & 0.728151 & -0.131346 & -0.090683 \\ 0.281798 & 0.050295 & -0.698391 & -0.655981 \\ -0.423756 & -0.393132 & 0.443861 & -0.684737 \\ -0.544701 & -0.559209 & -0.545881 & 0.304303\end{pmatrix}
\end{equation}
\par Simply, the overall percentage and cumulative percentage of eigenvalues are as follows.
\begin{table}[!hbp]
    \centering 
    \begin{tabular}{|c|c|c|}
\hline
Eigenvalue & Overall percentage & Cumulative percentage \\
\hline
$\lambda_1$=1.638729 & 0.409682 & 0.409682 \\
\hline
$\lambda_3$=1.089176 & 0.272294 & 0.681976 \\
\hline
$\lambda_4$=0.817484 & 0.204371 & 0.886347 \\
\hline
$\lambda_2$=0.454611 & 0.113653 & 1.000000 \\
\hline
\end{tabular}
\caption{Percentage}
\end{table}
\par Therefore, the cumulative percentage of $\lambda_1$ and $\lambda_3$ is 0.681976 colse to 0.7.
\par According to assumption 1, the main factors are natural gas and fuel oil, the corresponding eigenvalues are $\lambda_1$ and $\lambda_3$. Then we take out their corresponding eigenvectors from U, and we can obtain the factor loading matrix
\begin{equation}
    \centering
A=\begin{pmatrix} 0.856438 & -0.145463 \\ -0.657207 & -0.429761 \\ 0.673909 & -0.408171 \\ 0.138447 & 0.846592\end{pmatrix}
\end{equation}
the first row of A is
\begin{equation}
    \centering
\sqrt{\lambda_1} \times U[ ,1]
\end{equation}
the second row of A is
\begin{equation}
    \centering
\sqrt{\lambda_3} \times U[ ,3]
\end{equation}

% 4.2.1

\par According to the factor loading matrix A, we can get the factor analysis model is
\begin{equation}
    \centering
    \text{natural gas}=0.856438 \times f_1-0.145463 \times f_2
\end{equation}
\begin{equation}
    \centering
 \text{electricity}=-0.657207 \times f_1-0.429761 \times f_2
\end{equation}
\begin{equation}
    \centering
    \text{fuel oil}=0.673909 \times f_1-0.408171 \times f_2
\end{equation}
\begin{equation}
    \centering
    \text{coal coke}=0.138447 \times f_1+0.846592 \times f_2
\end{equation}
\par The definition of the weight of variables is\cite{5}
\begin{equation}
    \centering
w_i=\frac{ \theta_i }{ \sum_{i=1}^{4}\theta_i } \times 100\%
\end{equation}
\begin{equation}
    \centering
\theta_i=A[i,1] \times 0.409682+A[i,2] \times 0.272294+A[i,3] \times 0.204371+A[i,4] \times 0.113653
\end{equation}

\par Therefore, we can obtain the the weight of natural gas, electricity, fuel oil and coal coke.

\[
w_1=25.640737\%, w_2=25.775307\%, w_3=23.682713\%, w_4=24.901242\%.
\]

% 4.2.2
\subsubsection{Profiles of four kinds of energy}

\par From the point of the factor analysis model, we can easily find that $f_1$ is a positive factor for natural gas, fuel oil, coal coke, and a negative factor for electricity. Meanwhile, $f_2$ is a positive factor for coal coke, and a negative factor for natural gas, electricity, fuel oil. 
 So we can obtain the weights of variables of the four states in 2009 as follows.

\begin{table}[!htb]
    \centering
    \begin{tabular}{|c|c|c|c|c|}
        \hline state & natural gas & electricity & fuel oil & coal coke\\
        \hline CA & 25.64 & 25.77 & 23.68 & 24.90\\
        \hline TX & 29.99 & 32.87 & 12.18 & 24.96\\
        \hline NM & 35.43 & 20.15 & 10.55 & 33.87\\ 
        \hline AZ & 27.53 & 19.36 & 27.08 & 26.02\\
        \hline
    \end{tabular}
    \caption{The weights of variables of the four states in 2009}\label{tab:factors}
\end{table}

\par According to table \ref{tab:factors}, the contribution rates of the four kinds of energy are very close in California. But in New Mexico and Texas, natural, electricity, and coal coke are main energy, because the addition of their contribution rates is over 85\%. And in Arizona, the contribution rate of electricity is less than the other energy. 

% 4.3
\subsection{Renewable energy profile}

\par In the previous section, we have set up a good factor analysis model for energy classification. It dose help us build the profile of the states' energy, but it dosen't pay much attention to the difference between renewable energy and non-renewable energy. So in this part, we analyze the data set and give the profile of renewable energy in four states.

% 4.3.1
\subsubsection{Renewable energy usage percentage}
\par To get a preliminary understanding of the renewable energy usage in four states, we decide to get the proportion of renewable energy in total energy in the first step.

\par What's important is that, we find that there are some specific MSNs and formulas in the reference \cite{4} which can describe the usage of renewable energy clearly and precisely. MSN RETCB is a good example, which represents renewable energy sources total consumed in a state.
\par Then we search for the MSN in the given data and find every state have have this MSN. So this is indeed the representative MSN variable of the usage of renewable energy in every states. And then we find the TETCB MSN stands for the total energy consumed in a state. So we get the equation like this

\begin{equation}
    \eta_1=\frac{\text{RETCB}}{\text{TETCB}}
\end{equation}

\begin{figure}[h]%[!hptb] 
    \centering 
    \includegraphics[width=0.6\textwidth]{./Pic/1-2.png}
    % 图片标题
    \caption{The proportion of renewable energy in total energy in four states in the year of 2009.}
    \label{fig:1-2}  
\end{figure}
\par After calculating with the formula, we get the results of $\eta_1$, which can be seen in figure \ref{fig:1-2}

% 4.3.2
\subsubsection{Renewable energy usage in different sectors}
\par There are many factors that influence the usage of energy in a state. So it is reasonable to search for the situation of renewable energy usage in different sector in every state.
\par Similar to the previous section, we find that there are some MSNs that describe the renewable energy in certain sector and formulas that calculate for the value of it. Take transportation sector for example, MSN REACB stands for renewable energy sources consumed by the transportation sector, so the sector's percentage in renewable energy consumption can be calculated through the formula. After calculating with formulas, we get the results of $\eta_2$ , which can be seen in figure \ref{fig:1-3}.

\begin{equation}
    \eta_2=\frac{REACB}{RETCB} \times 100\%
\end{equation}

\begin{figure}[h]%[!hptb] 
    \centering 
    \includegraphics[width=0.6\textwidth]{./Pic/1-3.png}
    % 图片标题
    \caption{The proportion of different sector in renewable usage in four states in the year of 2009.}
    \label{fig:1-3}  
\end{figure}


% 5
\section{Profile Evolution Model \& Energy Usage Comparison Model}


\par We have set up a good model for energy profile evaluation (State Energy Analysis Model) to analyze the energy profile of each state at a given year. 
\par The profile evolution model takes \textbf{time factor} into consideration to get the energy profile evolution results of four states. Besides,
we will use a new model Analytic Hierarchy Process (AHP) to deal with the analysis of possible influential factors of similarities and differences among the four states.

% 5.1
\subsection{Profile Evolution Model}
\par On the basis of the energy profile model, we try to bring time factor into the model and get the evolution result of the profile. 

% 5.1.1
\subsubsection{Model improvement with time factor}
\par Along with how to bring time factor into the model, we have a basic solution of calculating the result of the model of each year and then displaying the result in a line chart. 

\par Take California as an example, we firstly use the factor analysis model to classify the energy into 4 kinds and divide them by 5 sectors, and according to step-counting principle, we get 4$\times$5 groups of basic energy profile unit. For everyone in the 20 units, we build a line chart to describe the trend as the year changes and put all of them into one diagram to have a more intuitive comparison, as can been seen in figure \ref{fig:part-2-CA}.  
\par Results of other states can been seen in the Appendix. % TODO: which appendix?

% TODO: 图片错位
\begin{figure}[H]%[!hptb] 
    \centering 
    \includegraphics[width=0.9\textwidth]{./Pic/part-2-CA.png}
    % 图片标题
    \caption{Evolution of energy profile of CA (1960-2009).}
    \label{fig:part-2-CA}  
\end{figure}

% 5.1.2
\subsubsection{Brief analysis of the evolution above}
\par There are indeed some advantages of this model evolution, which are as follows
\begin{itemize}
    \item Comparison between different classification of energy is well organized and can be seen directly and clearly.
    \item Different sectors have different impacts on energy usage, and this model can show the difference obviously.
    \item It is intelligible to see the trend of the energy usage changes over time.
    \item We can easily get to the conclusion that some sharp changes in the line chart are possibly caused by policy reform.  
\end{itemize}
\par However, we have to say it is undeniable that this evolution model has its disadvantages, which are as follows
\begin{itemize}
    \item It is hard to compare the influence of different factors in different states in this model, in other words, we can not see the comprehensive effect of all the factors together in energy usage influence.
    \item Governors are more care about the interstate difference on energy usage, but this model is not intuitively enough.
    \item The model doesn't pay much attention to the usage of renewable energy, which is an important part of our problem.
\end{itemize}
\par In order to improve the shortcomings, we build a new model, named Analytic Hierarchy Process (AHP), to satisfy the demand of the governors.


% 5.2
\subsection{Energy Usage Comparison Model}

\par With analytic hierarchy process(AHP) model, we can obtain scores of the usage of renewable energy of the four states, through which we can analysize the similarities and differences between the four states. 
% TODO: 使用摘要中的语句
% 5.2.1
\subsubsection{Analytic hierarchy and factor origin}

\par The first step to build the model is to find key factors influencing the state comparison. There is no doubt that the target comparing object is the usage of renewable energy and the bottom layer is four states. Then we find that there are mainly five sectors in which all the energy be produced or consumed, so these sectors are the best fatcors to describe the influence of energy usage similarities and differences between the four states.

\par The five sectors are transportation, commerce, electricity, industry, and resident. Resident naturally represents population, and transportation is closely linked with geography \cite{L1}. So our factors contain some of the possible influential factors of the similarities and differences described in the question, but they are more detailed and energy-concerned.

\par After getting the comparison results of total renewable energy usage, we start to focus on the per capita renewable energy usage. It is reasonable that different population of a state results in different energy usage. So we divide the renewable energy usage of every sectors by population and calculate its result of AHP.

% 5.2.2
\subsubsection{Establishment of AHP model}

\par For obtaining the renewable energy proportion of each sector, we need renewable energy total comsumption values of each sector which can be computed by formula in the MSN refercence:
\begin{align*}
& \text{transportation} = \text{EMACB} \\
& \text{commercial} = \text{EMCCB} + \text{GECCB} + \text{HYCCB} + \text{SOCCB} + \text{WWCCB} + \text{WYCCB} \\
& \text{electric power} = \text{HYEGB} + \text{GEEGB} + \text{SOEGB} + \text{WWEIB} + \text{WYEGB} \\
& \text{industrial} = \text{EMICB} + \text{EMLCB} + \text{GEICB} + \text{HYICB} + \text{SOICB} + \text{WWICB} + \text{WYICB} \\
& \text{residential} = \text{WDRCB} + \text{GERCB} + \text{SORCB} \\
\end{align*}

\par We suppose Y, which is the use level of renewable energy as the target, and $X_i$, which represents the five sectors as the standard. Therefore, we have the analytic hierarchy model based on total renewable energy usage as follows.\cite{6}

\begin{figure}[H] 
    \centering 
    \includegraphics[width=1.0\textwidth]{./Pic/AHP.pdf}
    \caption{AHP model}
\end{figure}

\par In order to analysize the closeness between each of the sectors and Y, we analysize the given data to obtain and then use the total of the renewable energy usage of each sector. According to the calculation method of the pairwise comparison matrix, we can obtain C, which is the pairwise comparison matrix of the five sectors under Y. Simply, we can calculate $\lambda_m$(C), which is the maximum of the eigenvalues of C. Here, n is equal to 4 and RI is equal to 1.12.\cite{6} If C can not pass the check equation of consistency:
\begin{equation}
    \centering
    CI=\frac{ \lambda_m(C)-n }{ n-1 }
\end{equation}
\begin{equation}
        \centering
    CR=\frac{ CI }{ RI }<0.1 
\end{equation}

\par We will recalculate C until C pass the check of consistency. Then we calculate $\beta$, the eigenvector of $\lambda_m$(C), which meets the conditions that the sum of the coordinates of the eigenvector is 1 and the coordinates is greater than 0, called weight vector.
\par Similarly, we can also obtain $B_i$, which is the pairwise comparison matrix of the four states under $X_i$, and $\beta_i$, which is the weight vector of $X_i$.
\par We define W as follows: % TODO: follows?
\begin{equation}
    \centering
U=\begin{pmatrix} \beta_1 & \beta_2 & \beta_3 & \beta_4\end{pmatrix}
\end{equation}
\par Finally, we can obtain the scores based on the total energy usage of the four states:
\begin{equation}
    \centering
    CA=\sum_{i=1}^{5}\beta[i] \times U[i,1]
\end{equation}
\begin{equation}
    \centering
    TX=\sum_{i=1}^{5}\beta[i] \times U[i,2]
\end{equation}
\begin{equation}
    \centering
    NM=\sum_{i=1}^{5}\beta[i] \times U[i,3]
\end{equation}
\begin{equation}
    \centering
    AZ=\sum_{i=1}^{5}\beta[i] \times U[i,4]
\end{equation}

% 5.2.3
\subsubsection{Calculation results of AHP model}
\par Firstly, we divide the given data by year, and calculate the the scores of the four states in 2009 as follows:
\[
    \text{CA}=57.47503267\%, \text{TX}=36.64666161\%, \text{NM}=0.575336357\%, \text{AZ}=5.302969359\%
\]
Clearly, the score of California and Texas is much greater than New Mexico and Arizona.
Then we calculate the scores of the four states in every year, and draw a line chart as follows.
\begin{figure}[!hptb] 
    \centering 
    \includegraphics[width=0.8\textwidth]{./Pic/B-level-predict.png}
    \caption{Scores based on the total renewable energy usage from 1960 to 2009}
    \label{fig:B-level-predict}
\end{figure}

% 5.2.4
\subsubsection{Result of per capita usage}
\par In order to have a more detailed and more comprehensive comparison of the similaritiesand differences of the states, we hold the view that it is not enough to compare the proportionof the renewable energy usage among states, sector usage difference within the state is also an important comparing factor.
\par When we talk about the analytic hierarchy model based on per capita renewable energyusage, we just need to change CA, TX, NM, AZ to CA, TX, NM, AZ divided by the total population of each state. Therefore, similarly we can obtain the scores based on the per capita energy usage of the four states in each year from 1960 to 2009, and draw the line chart as follows.
\begin{figure}[H] 
    \centering 
    \includegraphics[width=0.8\textwidth]{./Pic/B-level-percapita.png}
    \caption{Scores based on the per capita renewable energy usage from 1960 to 2009}
    \label{fig:B-level-percapita}
\end{figure}

\subsection{Similarities and differences analysis}



\par We have taken the five sectors' influence into consideration in the AHP model to get the renewable energy usage proportion, and it works well when comparing the sector influence among four states, but it can not describe the proportion of different sectors in a state. So we calculate the proportion of renewable energy usage of different sectors in every state in every year from 1960 to 2009 and obtain figure \ref{fig:B-percent}.
\begin{figure}[H]%[!hptb] 
    \centering 
    \includegraphics[width=0.7\textwidth]{./Pic/B-percent.png}
    % 图片标题
    \caption{The proportion of renewable energy usage of different sectors of four states in 2009}
    \label{fig:B-percent}  
\end{figure}

\par We have take the five sectors' influence into consideration in the AHP model to get the renewable energy usage proportion, and it works well when comparing the sector influence among four states, but it can not describe the proportion of different sectors in a state. So we calculate the proportion of renewable energy usage of different sectors in every state in every year from 1960 to 2009 and get figure \ref{fig:B-percent}. According to figure \ref{fig:B-level-predict} and figure \ref{fig:B-percent} we can conclude the similarities and differences.

% TODO: which diffs?

% 5.3.1
\subsubsection{Differences among the four states}

\begin{itemize}
    \item Different states account for different proportion of the total renewable energy usage. California always accounts for the most while New Mexico always accounts for the least.
    \item Different states have different level of the pre capita renewable energy usage. California and Texas always accounts for the least while New Mexico and Arizona always accounts for the most.
    \item The fluctuation of the proportion of New Mexico is not so obvious, but other three states have changed a lot.
    \item After 2000, the score based on the total renewable energy of Arizona has a steady trend of declining, while other states keep fluctuating.
    \item Different state has different major usage of renewable energy. California and Arizona use renewable energy mainly in electric power, New Mexico mainly in resident sector and Texas mainly in industry.
    \item The distribution of tatal renewable energy usage has changed a lot in all of the four states, but the changing trend of each state is different from others.
    
\end{itemize}

% 5.3.2
\subsubsection{Similarities among the four states}

\begin{itemize}
    \item All of the states can be easily affected by policy change, which can be infered from the sharp rise and fall in the figure.
    \item All of the states have different distribution of total renewable energy usage.
    \item All of the states have little renewable energy usage in commercial sector.
    \item All of the states have a trend of increasing proportion of transportation, while it was almost zero at first.
    \item All of the states do not have a huge change in the level of per capita renewable energy usage from 1960 to 2009.
\end{itemize}


% 6
\section{Best Profile Analysis Model}

\par The governors are interested at forming a realistic new energy compact focused on increased usage of cleaner, renewable energy sources. But before building the compact, it is of great significance to compare the renewable energy use profile among four states and get the best profile to encourage states to accelerate the development of renewable energy.

\par We have compared the profile of four states in previous section a lot, so in this part, we foucs on setting up a series of criteria and choosing a best profile of them. 

\subsection{Best profile in renewable energy usage amount}

\par Based on the result of Problem B, we can get the scores of the importance of renewable energy usage of different states, which actually stands for the state's proportion of renewable energy usage in all states under the influence of five sectors. On this occasion, we can get the biggest proportion, which represents the state that uses the most renewable energy in the year.
\par Let $\phi$ represents the proportion of each state in 2009, then we can get the result as follows from the calculation of Problem B

\begin{equation}
    \phi_{CA} = 55.65557290,~~ \phi_{TX} = 38.04528274,~~ \phi_{NM} = 0.99719276,~~ \phi_{AZ} = 5.30195160
\end{equation}

\par $\phi_{max} = \phi_{CA}$, so the biggest proportion is $\phi_{CA}$, which means that \textbf{California} has the biggest energy usage amount in 2009.

\par But other governors may have different opinions, they may hold the view that it is unfair since California has the biggest population. On this occasion, we will take the result of per capita usage for analysis. Let $\Phi$ represents the per capita renewable energy usage proportion of each state in 2009, then we can get the data

\begin{equation}
    \Phi_{CA} = 4.05334714,~~ \Phi_{TX} = 8.00804230,~~ \Phi_{NM} = 44.55708957,~~ \Phi_{AZ} = 43.38152098
\end{equation}
\par $\Phi_{max} = \Phi_{NM}$, so the biggest per capita proportion is $\Phi_{NM}$, which means that \textbf{New Mexico} has the biggest per capita energy usage amount in 2009.

\subsection{Best profile in increase of renewable energy usage}

\par Besides total amount of renewable energy usage, we think that it is also indispensable to take usage increase rate into consideration, because increase rate can reflect a state's growing concern about the renewable energy and can represent the progress in renewable energy usage. It is also an important factor when judging the "best" profile.

\par The best description of the increase of renewable energy usage is its gradient in the line chart. Let $\kappa$ be the gradient of every state, then 

\begin{equation}
    \kappa = \frac{\Delta \phi}{\Delta year}
\end{equation}

\par Then we bring data into the formula and get results
\begin{equation}
    \kappa_{CA} = -4.36938978,~~ \kappa_{TX} = 6.07795471,~~ \kappa_{NM} = -0.10693533,~~ \kappa_{AZ} = -1.60162960
\end{equation}
\par $\kappa_{max} = \kappa_{TX}$, so \textbf{Texas} has the biggest increase rate in 2009.

\par Similarly, we take population into consideration and get the gradient of the  proportion of per capita renewable energy usage. Let $\mu$ be the gradient of every state, then 
\begin{equation}
    \mu = \frac{\Delta \Phi}{\Delta year}
\end{equation}

\par Then we bring data into the formula and get results
\begin{equation}
    \mu_{CA} = -0.23615508,~~ \mu_{TX} = -1.21046745,~~ \mu_{NM} = 0.93103726,~~ \mu_{AZ} = 0.51558527
\end{equation}
\par $\mu_{max} = \mu_{NM}$, so \textbf{New Mexico} has the biggest per capita increase rate in 2009.

\subsection{Best profile choice}
\par In summary, there are a variety of perspectives to choose a best profile among the four state. Different states have their own best performance in different perspectives, but after taking various factors into consideration,  \textbf{New Mexico} has the most best performance in renewable energy usage.
So the best profile of 2009 belongs to \textbf{New Mexico}.



% \begin{figure}[!h]
%   \centering %使得插入的照片居中显示
%   \begin{minipage}[t]{.49\linewidth}
%    % \framebox{Text}
%   \includegraphics[width=1\textwidth]{./Pic/P3_.png}
%   \caption{The Network Effciency in Different Years (1990-2015)}
%   \end{minipage}
%   \begin{minipage}[t]{.49\linewidth}
%   \includegraphics[width=1\textwidth]{./Pic/P3.png}
%    %\framebox{Text}
%   \caption{The Proportion of Different Information Dissemination Media Pie Chart (around 2050).}
%   \end{minipage}
% \end{figure}





\section{Part \uppercase\expandafter{\romannumeral1} Problem D : Prediction of energy profile in 2025 and 2050}
\par Based on the results of problem B, we build regression equations, which show the relationships between the scores of the four states and years, so that we can predict the socres of the four states in 2025 and 2050.
\subsection{Regression analysis model}  
\subsubsection{Regression equations}
\par In problem D, the policy about energy of each state will not change until 2050. At first, we consider the levels of total renewable energy usage of the four states.  According to Figure \ref{fig:B-level-predict}, from 1995 to 2009, the scores of New Mexico and Arizona have a consistent trend, we consider it as the absence of any policy changes. Meanwhile, for California and Texas, their scores have a sudden chaneg after 2005, we consider it is because of policy change. Therefore, we use the scores of New Mexico and Arizona from 1995 to 2009, and the socres  of California and Texas from 2005 to 2009 to build the regression equations.
\par The regression equations are
\begin{equation}
    \centering
CA=e^{-0.34649 \times year+697.98712}+50
\end{equation}
\begin{equation}
    \centering
TX=100-CA-NM-AZ
\end{equation}
\begin{equation}
    \centering
NM=0.05263 \times year-104.64807
\end{equation}
\begin{equation}
    \centering
AZ=e^{-0.10513 \times year+213.24602}
\end{equation}
\par The $R^{2}$ of each regression are greater than 0.68, even over 0.8. Therefore, according to assumption 3, all of the effects of the regressions are good.
\par The Pr of F-test and t-test are all less than 0.05. Therefore, according to assumption 2, all of the regressions are effective and the independences between the parameters are very good.

\subsubsection{Scores of the four states in future}
\par According to the regression equations, we can obtain the socres based on the total renewable energy usage of the four states in 2025 and 2050 are as follows.
\begin{table}[!hbp]
    \centering 
    \begin{tabular}{|c|c|c|}
\hline
State & Score-Total in 2025 & Score-Total in 2050 \\
\hline
California & 50.04186\% & 50.00001\% \\
\hline
Texas & 46.76163\% & 47.29053\% \\
\hline
New Mexico & 1.59588\% & 2.56588\% \\
\hline
Arizona & 1.60063\% & 0.14358\% \\
\hline
\end{tabular}
\caption{Scores of four states in 2025 and 2050}
\end{table}
\par Besides, we draw the graph of the regression equations as follows.
\begin{figure}[h]%[!hptb] 
    \centering 
    \includegraphics[width=0.7\textwidth]{./Pic/D-scores.png}
    \caption{Scores based on total renewable energy of the four states from 1960 to 2050}
    \label{fig:D-scores}  
\end{figure}
\par Besides, we also need to consider the levels of per capita renewable energy usage of the four states. According to figure \ref{fig:B-level-percapita}, we can find that each of the score based on per capita renewable energy usage are in a approximate region.
\[
CA:[1,10], TX:[1,10], NM:[43,50], AZ:[40,50]
\]

\subsection{Analysis of the results}
\subsubsection{Level of total renewable energy usage}
\par From 2025 to 2050, based on the total renewable energy usage, socre of California is very close to 50.00000 and continues to decline, score of Arizona begins to be less than the socre of New Mexico and is close to 0 in 2050. On the contrary, score of Texas continues to grow but is not greater than the score of California, and the score of New Mexico grows with a slow rate.
\par Therefore, comparing to other states, the levels of total renewable energy usage of the four states are as follows.
\begin{itemize}
    \item California: declines but still is the best.
    \item Texas: keeps growing and gets colse to California.
    \item New Mexico: get better than Arizona and also keeps growing.
    \item Arizona: is not good and need to change.
\end{itemize}
\subsubsection{Level of pre capita renewable energy usage}
\par The socres based on the per capita renewable energy usage do not have a huge change, they are still in the region.
\par Therefore, comparing to other states, the level of per capita renewable energy usage are as follows.
\begin{itemize}
    \item California and Texas: are still the lowest.
    \item New Mexico and Arizona: are still the highest.
\end{itemize}












\section{Part \uppercase\expandafter{\romannumeral2}, Problem A}
\par In the first question, we set up a model of public opinion dissemination. In this chapter, we will analyze the factors that influence the spread of public opinion. At the same time, a real society is analyzed.
\subsection{Influence Factors of Public Opinion Disseminate}
\par There are many factors that affect the spread of public opinion, and we choose the following ~$4$~ factors as the main factors.


\begin{enumerate}%[(1)]
\renewcommand{\labelenumi}{(\theenumi)}
    \item Information value

    The formula for calculating the value of information has been given in the first question. Information value can influence not only the speed of information dissemination, but also the public opinion. value of information is high, the probability of people receiving information and the degree of acceptance will be improved.


    \item People's initial opinions and prejudices

    Psychologists believe that humans are "cognitive misers"\cite{RE}. People always simplify the cognitive process as much as possible. It is very difficult for people to change their ideas once they have produced a view or bias towards an event.
    \item Form and source of information

    The form and source of the information can impact the value of the message, which would also affect the public opinion. The evaluation formula of information's source and form have been given in the previous paper.


    \item Network topology structure and strength

    We will use several indexes of network topology to measure the efficiency of information dissemination
    The topology or strength of the information network in a region, country, or worldwide is depended on the three index: Average Degree ~$AD$~, Average Path Length ~$L$~, and Average Path Length ~$L$~.
\end{enumerate}
\par Taking into account the indexes of the public opinion Dissemination are difficult to be quantified, for simplicity, we do not consider quantitative analysis of the spread of public opinion model, we only evaluate it subjectively.

\subsection{Analysis One Historical Event based on Public Opinion Dissemination Model}
\par In order to express more vivid description of the model we build, we look for a historical event, and analyze the historical events on the basis of our model.



\begin{figure}[!h]
  \centering %使得插入的照片居中显示
  \begin{minipage}[t]{.49\linewidth}
   % \framebox{Text}
  \includegraphics[width=1\textwidth]{./Pic/obama.png}
  \caption{President Obama gets his own Twitter account.}
  \end{minipage}
  \begin{minipage}[t]{.49\linewidth}
  \includegraphics[width=1\textwidth]{./Pic/p5.jpg}
   %\framebox{Text}
  \caption{Graph of President Obama's follower growth.\cite{RG}}
  \label{fig:p5}
  \end{minipage}
\end{figure}
\par "Hello, Twitter! It's Barack. Really! Six years in, they're finally giving me my own account.". After President Obama opened his Facebook account, he opened his Twitter account on May 18, 2015.
\par From Figure \ref{fig:p5}, President Obama's followers has been growing rapidly over time. In our model, President Obama is one of the \textbf{\emph{Opinion Leader}}, the news that he has released is a hot and authoritative topic of concern.
\par The analysis results of this case are very consistent with our expectations, therefore, this example verifies the accuracy of our model.




\section{Strengths and Weaknesses}

\paragraph{Strengths}
\text{\\}
\begin{enumerate}%[(1)]
\renewcommand{\labelenumi}{(\theenumi)}
    \item In Problem A, we used factor analysis to obtain the energy profile of the four kines of energy, besides the energy profile of renewable energy.
    \item In Problem B, we considered the influence of the total population of each state, which made our model base on the situation of each state fully.
    \item There are many different factors that affect the complicated problem. We classify the influencing factors and put forward the quantitative indexes. At the same time, we use of examples to test the quantitative indicators which are screened out. Only in this way, can we select out the well-founded indicators.
\end{enumerate}


\paragraph{Weaknesses}
\begin{enumerate}%[(1)]
\renewcommand{\labelenumi}{(\theenumi)}
    \item In factor analysis, we supposed that the corresponding valuables as the main factors when cumulative percentage is over 0.65, which may not enough.
    \item Due to the amount of the data of scores is not enough, so that thre results of regressions may not match future very much.
\end{enumerate}




\newpage%另起一页书写正文
\thispagestyle{empty}%本页不遍页码
%以下是参考文献
\phantomsection%生成该页的链接
\addcontentsline{toc}{section}{\refname}
\begin{thebibliography}{}
%
% 使用指令\bibitem 构造一条参考文献.
% 具体构造方式,参考以下参考文献格式说明以及示例
% 应尽可能使用英文格式
%
\bibitem{1}
Wikipedia, Interstate compact. \url{https://en.wikipedia.org/wiki/Interstate_compact}. 2018

\bibitem{2}
Wikipedia, Factor analysis. \url{https://en.wikipedia.org/wiki/Factor_analysis#Type_of_factor_analysis}. 2018

\bibitem{3}
U.S. Energy Information Administration, State Energy Data System (SEDS): 1960-2015 (complete). \url{https://www.eia.gov/state/seds/seds-data-complete.php}. 2017

\bibitem{4}
U.S. Energy Information Administration, State Energy Data System (SEDS): 1960-2015 (complete). \url{https://www.eia.gov/state/seds/sep_use/notes/use_a.pdf}. 2017.

\bibitem{5}
Tang Mengling, Wang zhanling, and Li Zhijian. Using Model of Fcator Analysis to Calculate Weight and to Evaluate Water Quality[J]. Xingtai Vocational and Technical College, 2005.

\bibitem{RC}
U.S. Energy Information Administration, Renewable Energy Explained. \url{https://www.eia.gov/energyexplained/?page=renewable_home}. 2017


\bibitem{6}
Thomas L. Satty. How to make a decision: The Analytic Hierarchy Process[J]. University of Pittsburgh, 1990.

\bibitem{L1}
Wikipedia, Transport geography. \url{https://en.wikipedia.org/wiki/Transport_geography}. 2017


\bibitem{RF}
Pewresearch, State of the News Media 2012. \url{http://www.pewresearch.org/2012/03/19/state-of-the-news-media-2012/}. 2012

\bibitem{RK}
International Telecommunication Union, ICT Data and Statistics (IDS). \url{http://www.itu.int/ITU-D/ict/statistics/explorer/index.html}. 2016

\bibitem{RH}
Wikipedia, Quantum information science. \url{https://en.wikipedia.org/wiki/Quantum_information_science}. 2016
% etc
\end{thebibliography}


\newpage%另起一页书写
\begin{appendices}  
    \section*{Simulation Code}  
    some text in Appendix A  
    \section*{Other figures}

\begin{figure}[h]%[!hptb] 
    \centering 
    \includegraphics[width=0.9\textwidth]{./Pic/part-2-AZ.png}
    % 图片标题
    \caption{Evolution of energy profile of AZ (1960-2009).}
    \label{fig:part-2-AZ}  
\end{figure}

\begin{figure}[h]%[!hptb] 
    \centering 
    \includegraphics[width=0.9\textwidth]{./Pic/part-2-NM.png}
    % 图片标题
    \caption{Evolution of energy profile of NM (1960-2009).}
    \label{fig:part-2-NM}  
\end{figure}

\begin{figure}[h]%[!hptb] 
    \centering 
    \includegraphics[width=0.9\textwidth]{./Pic/part-2-TX.png}
    % 图片标题
    \caption{Evolution of energy profile of TX (1960-2009).}
    \label{fig:part-2-TX}  
\end{figure}
    some text in Appendix B  
\end{appendices}  


\end{document}
